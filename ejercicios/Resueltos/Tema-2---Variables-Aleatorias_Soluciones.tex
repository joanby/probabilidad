% Options for packages loaded elsewhere
\PassOptionsToPackage{unicode}{hyperref}
\PassOptionsToPackage{hyphens}{url}
\PassOptionsToPackage{dvipsnames,svgnames*,x11names*}{xcolor}
%
\documentclass[
]{article}
\usepackage{lmodern}
\usepackage{amssymb,amsmath}
\usepackage{ifxetex,ifluatex}
\ifnum 0\ifxetex 1\fi\ifluatex 1\fi=0 % if pdftex
  \usepackage[T1]{fontenc}
  \usepackage[utf8]{inputenc}
  \usepackage{textcomp} % provide euro and other symbols
\else % if luatex or xetex
  \usepackage{unicode-math}
  \defaultfontfeatures{Scale=MatchLowercase}
  \defaultfontfeatures[\rmfamily]{Ligatures=TeX,Scale=1}
\fi
% Use upquote if available, for straight quotes in verbatim environments
\IfFileExists{upquote.sty}{\usepackage{upquote}}{}
\IfFileExists{microtype.sty}{% use microtype if available
  \usepackage[]{microtype}
  \UseMicrotypeSet[protrusion]{basicmath} % disable protrusion for tt fonts
}{}
\makeatletter
\@ifundefined{KOMAClassName}{% if non-KOMA class
  \IfFileExists{parskip.sty}{%
    \usepackage{parskip}
  }{% else
    \setlength{\parindent}{0pt}
    \setlength{\parskip}{6pt plus 2pt minus 1pt}}
}{% if KOMA class
  \KOMAoptions{parskip=half}}
\makeatother
\usepackage{xcolor}
\IfFileExists{xurl.sty}{\usepackage{xurl}}{} % add URL line breaks if available
\IfFileExists{bookmark.sty}{\usepackage{bookmark}}{\usepackage{hyperref}}
\hypersetup{
  pdftitle={Ejercicios Tema 2 - Variables aleatorias},
  pdfauthor={Ricardo Alberich, Juan Gabriel Gomila y Arnau Mir},
  colorlinks=true,
  linkcolor=Maroon,
  filecolor=Maroon,
  citecolor=Blue,
  urlcolor=Blue,
  pdfcreator={LaTeX via pandoc}}
\urlstyle{same} % disable monospaced font for URLs
\usepackage[margin=1in]{geometry}
\usepackage{color}
\usepackage{fancyvrb}
\newcommand{\VerbBar}{|}
\newcommand{\VERB}{\Verb[commandchars=\\\{\}]}
\DefineVerbatimEnvironment{Highlighting}{Verbatim}{commandchars=\\\{\}}
% Add ',fontsize=\small' for more characters per line
\usepackage{framed}
\definecolor{shadecolor}{RGB}{248,248,248}
\newenvironment{Shaded}{\begin{snugshade}}{\end{snugshade}}
\newcommand{\AlertTok}[1]{\textcolor[rgb]{0.94,0.16,0.16}{#1}}
\newcommand{\AnnotationTok}[1]{\textcolor[rgb]{0.56,0.35,0.01}{\textbf{\textit{#1}}}}
\newcommand{\AttributeTok}[1]{\textcolor[rgb]{0.77,0.63,0.00}{#1}}
\newcommand{\BaseNTok}[1]{\textcolor[rgb]{0.00,0.00,0.81}{#1}}
\newcommand{\BuiltInTok}[1]{#1}
\newcommand{\CharTok}[1]{\textcolor[rgb]{0.31,0.60,0.02}{#1}}
\newcommand{\CommentTok}[1]{\textcolor[rgb]{0.56,0.35,0.01}{\textit{#1}}}
\newcommand{\CommentVarTok}[1]{\textcolor[rgb]{0.56,0.35,0.01}{\textbf{\textit{#1}}}}
\newcommand{\ConstantTok}[1]{\textcolor[rgb]{0.00,0.00,0.00}{#1}}
\newcommand{\ControlFlowTok}[1]{\textcolor[rgb]{0.13,0.29,0.53}{\textbf{#1}}}
\newcommand{\DataTypeTok}[1]{\textcolor[rgb]{0.13,0.29,0.53}{#1}}
\newcommand{\DecValTok}[1]{\textcolor[rgb]{0.00,0.00,0.81}{#1}}
\newcommand{\DocumentationTok}[1]{\textcolor[rgb]{0.56,0.35,0.01}{\textbf{\textit{#1}}}}
\newcommand{\ErrorTok}[1]{\textcolor[rgb]{0.64,0.00,0.00}{\textbf{#1}}}
\newcommand{\ExtensionTok}[1]{#1}
\newcommand{\FloatTok}[1]{\textcolor[rgb]{0.00,0.00,0.81}{#1}}
\newcommand{\FunctionTok}[1]{\textcolor[rgb]{0.00,0.00,0.00}{#1}}
\newcommand{\ImportTok}[1]{#1}
\newcommand{\InformationTok}[1]{\textcolor[rgb]{0.56,0.35,0.01}{\textbf{\textit{#1}}}}
\newcommand{\KeywordTok}[1]{\textcolor[rgb]{0.13,0.29,0.53}{\textbf{#1}}}
\newcommand{\NormalTok}[1]{#1}
\newcommand{\OperatorTok}[1]{\textcolor[rgb]{0.81,0.36,0.00}{\textbf{#1}}}
\newcommand{\OtherTok}[1]{\textcolor[rgb]{0.56,0.35,0.01}{#1}}
\newcommand{\PreprocessorTok}[1]{\textcolor[rgb]{0.56,0.35,0.01}{\textit{#1}}}
\newcommand{\RegionMarkerTok}[1]{#1}
\newcommand{\SpecialCharTok}[1]{\textcolor[rgb]{0.00,0.00,0.00}{#1}}
\newcommand{\SpecialStringTok}[1]{\textcolor[rgb]{0.31,0.60,0.02}{#1}}
\newcommand{\StringTok}[1]{\textcolor[rgb]{0.31,0.60,0.02}{#1}}
\newcommand{\VariableTok}[1]{\textcolor[rgb]{0.00,0.00,0.00}{#1}}
\newcommand{\VerbatimStringTok}[1]{\textcolor[rgb]{0.31,0.60,0.02}{#1}}
\newcommand{\WarningTok}[1]{\textcolor[rgb]{0.56,0.35,0.01}{\textbf{\textit{#1}}}}
\usepackage{longtable,booktabs}
% Correct order of tables after \paragraph or \subparagraph
\usepackage{etoolbox}
\makeatletter
\patchcmd\longtable{\par}{\if@noskipsec\mbox{}\fi\par}{}{}
\makeatother
% Allow footnotes in longtable head/foot
\IfFileExists{footnotehyper.sty}{\usepackage{footnotehyper}}{\usepackage{footnote}}
\makesavenoteenv{longtable}
\usepackage{graphicx}
\makeatletter
\def\maxwidth{\ifdim\Gin@nat@width>\linewidth\linewidth\else\Gin@nat@width\fi}
\def\maxheight{\ifdim\Gin@nat@height>\textheight\textheight\else\Gin@nat@height\fi}
\makeatother
% Scale images if necessary, so that they will not overflow the page
% margins by default, and it is still possible to overwrite the defaults
% using explicit options in \includegraphics[width, height, ...]{}
\setkeys{Gin}{width=\maxwidth,height=\maxheight,keepaspectratio}
% Set default figure placement to htbp
\makeatletter
\def\fps@figure{htbp}
\makeatother
\setlength{\emergencystretch}{3em} % prevent overfull lines
\providecommand{\tightlist}{%
  \setlength{\itemsep}{0pt}\setlength{\parskip}{0pt}}
\setcounter{secnumdepth}{5}

\title{Ejercicios Tema 2 - Variables aleatorias}
\author{Ricardo Alberich, Juan Gabriel Gomila y Arnau Mir}
\date{Curso de Probabilidad y Variables Aleatorias con R y Python}

\begin{document}
\maketitle

{
\hypersetup{linkcolor=blue}
\setcounter{tocdepth}{4}
\tableofcontents
}
\hypertarget{variables-aleatorias-discretas}{%
\section{Variables aleatorias
discretas}\label{variables-aleatorias-discretas}}

\hypertarget{problema-1.}{%
\subsection{Problema 1.}\label{problema-1.}}

Hay 10 estudiantes inscritos en una clase de Estadística, de entre los
cuales 3 tienen 19 años, 4 tienen 20 años, 1 tiene 21 años, 1 tiene 24
años y 1 tiene 26 años. De esta clase se seleccionan dos estudiantes sin
reposición. Sea \(X\) la edad media de los dos estudiantes
seleccionados. Hallar la función de probabilidad para \(X\).

\hypertarget{soluciuxf3n}{%
\subsubsection{Solución}\label{soluciuxf3n}}

\textless\textless\textless\textless\textless\textless\textless{} HEAD
Los valores que puede alcanzar \(X\) son los siguientes:

\begin{itemize}
\tightlist
\item
  \(X=19\) si se eligen los dos estudiantes de 19 años.
\item
  \(X=19.5\) si se elige un estudiante de 19 años y uno de 20 años.
\item
  \(X=20\) si se eligen los dos estudiantes de 20 años o un estudiante
  de 19 años y el otro de 21 años.
\item
  \(X=20.5\) si se elige un estudiante de 20 años y otro de 21 años.
\item
  \(X=21.5\) si se elige un estudiante de 19 años y otro de 24 años.
\item
  \(X=22\) si se elige un estudiante de 20 años y otro de 24 años.
\item
  \(X=22.5\) si se elige un estudiante de 19 años y otro de 26 años o un
  estudiante de 21 años y otro de 24 años.
\item
  \(X=23\) si se elige un estudiante de 20 años y otro de 26 años.
\item
  \(X=23.5\) si se elige un estudiante de 21 años y otro de 26 años.
\item
  \(X=25\) si se elige un estudiante de 24 años y otro de 26 años.
\end{itemize}

La función de probabilidad de \(X\) es la siguiente: \[
P_X(x)=P(X=x)=\frac{\mbox{Casos favorables}}{\mbox{Casos posibles}}=
\left\{\begin{array}{ll}
\frac{\binom{3}{2}}{\binom{10}{2}}=\frac{3}{45}=0.0666667, & \mbox{si } x=19,
 \\[0.25cm]
\frac{3\cdot 4}{\binom{10}{2}}=\frac{12}{45}=0.2666667, & \mbox{si } x=19.5,
 \\[0.25cm]
 \frac{\binom{4}{2}}{\binom{10}{2}}+\frac{3}{\binom{10}{2}}=\frac{6}{45}+\frac{3}{45}=0.2, & \mbox{si } x=20,
 \\[0.25cm]
 \frac{4\cdot 1}{\binom{10}{2}}=\frac{4}{45}=0.0888889, & \mbox{si } x=20.5,
 \\[0.25cm]
 \frac{3\cdot 1}{\binom{10}{2}}=\frac{3}{45}=0.0666667, & \mbox{si } x=21.5,
 \\[0.25cm]
 \frac{4\cdot 1}{\binom{10}{2}}=\frac{4}{45}=0.0888889, & \mbox{si } x=22,
 \\[0.25cm]
 \frac{3}{\binom{10}{2}}+\frac{1}{\binom{10}{2}}=\frac{3}{45}+\frac{1}{45}=0.0888889, & \mbox{si } x=22.5,
 \\[0.25cm]
 \frac{4\cdot 1}{\binom{10}{2}}=\frac{4}{45}=0.0888889, & \mbox{si } x=23,
 \\[0.25cm]
 \frac{1}{\binom{10}{2}}=\frac{1}{45}=0.0222222, & \mbox{si } x=23.5,
 \\[0.25cm]
 \frac{1}{\binom{10}{2}}=\frac{1}{45}=0.0222222, & \mbox{si } x=25,
 \\[0.25cm]
0, & \mbox{ en cualquier otro caso},
\end{array}\right.
\]

\begin{Shaded}
\begin{Highlighting}[]
\NormalTok{edades=}\KeywordTok{c}\NormalTok{(}\DecValTok{19}\NormalTok{,}\DecValTok{19}\NormalTok{,}\DecValTok{19}\NormalTok{,}\DecValTok{20}\NormalTok{,}\DecValTok{20}\NormalTok{,}\DecValTok{20}\NormalTok{,}\DecValTok{20}\NormalTok{,}\DecValTok{21}\NormalTok{,}\DecValTok{24}\NormalTok{,}\DecValTok{26}\NormalTok{)}
\NormalTok{edades}
\end{Highlighting}
\end{Shaded}

\begin{verbatim}
##  [1] 19 19 19 20 20 20 20 21 24 26
\end{verbatim}

\begin{Shaded}
\begin{Highlighting}[]
\NormalTok{casos=gtools}\OperatorTok{::}\KeywordTok{permutations}\NormalTok{(}\DecValTok{10}\NormalTok{,}\DataTypeTok{r=}\DecValTok{2}\NormalTok{)}
\NormalTok{casos}
\end{Highlighting}
\end{Shaded}

\begin{verbatim}
##       [,1] [,2]
##  [1,]    1    2
##  [2,]    1    3
##  [3,]    1    4
##  [4,]    1    5
##  [5,]    1    6
##  [6,]    1    7
##  [7,]    1    8
##  [8,]    1    9
##  [9,]    1   10
## [10,]    2    1
## [11,]    2    3
## [12,]    2    4
## [13,]    2    5
## [14,]    2    6
## [15,]    2    7
## [16,]    2    8
## [17,]    2    9
## [18,]    2   10
## [19,]    3    1
## [20,]    3    2
## [21,]    3    4
## [22,]    3    5
## [23,]    3    6
## [24,]    3    7
## [25,]    3    8
## [26,]    3    9
## [27,]    3   10
## [28,]    4    1
## [29,]    4    2
## [30,]    4    3
## [31,]    4    5
## [32,]    4    6
## [33,]    4    7
## [34,]    4    8
## [35,]    4    9
## [36,]    4   10
## [37,]    5    1
## [38,]    5    2
## [39,]    5    3
## [40,]    5    4
## [41,]    5    6
## [42,]    5    7
## [43,]    5    8
## [44,]    5    9
## [45,]    5   10
## [46,]    6    1
## [47,]    6    2
## [48,]    6    3
## [49,]    6    4
## [50,]    6    5
## [51,]    6    7
## [52,]    6    8
## [53,]    6    9
## [54,]    6   10
## [55,]    7    1
## [56,]    7    2
## [57,]    7    3
## [58,]    7    4
## [59,]    7    5
## [60,]    7    6
## [61,]    7    8
## [62,]    7    9
## [63,]    7   10
## [64,]    8    1
## [65,]    8    2
## [66,]    8    3
## [67,]    8    4
## [68,]    8    5
## [69,]    8    6
## [70,]    8    7
## [71,]    8    9
## [72,]    8   10
## [73,]    9    1
## [74,]    9    2
## [75,]    9    3
## [76,]    9    4
## [77,]    9    5
## [78,]    9    6
## [79,]    9    7
## [80,]    9    8
## [81,]    9   10
## [82,]   10    1
## [83,]   10    2
## [84,]   10    3
## [85,]   10    4
## [86,]   10    5
## [87,]   10    6
## [88,]   10    7
## [89,]   10    8
## [90,]   10    9
\end{verbatim}

\begin{Shaded}
\begin{Highlighting}[]
\NormalTok{casos\_edad=}\KeywordTok{data.frame}\NormalTok{(}\DataTypeTok{uno=}\NormalTok{edades[casos[,}\DecValTok{1}\NormalTok{]],}
                      \DataTypeTok{dos=}\NormalTok{edades[casos[,}\DecValTok{2}\NormalTok{]])}
\NormalTok{casos\_edad}\OperatorTok{$}\NormalTok{media=}\KeywordTok{apply}\NormalTok{(casos\_edad,}\DecValTok{1}\NormalTok{,mean)}
\NormalTok{casos\_edad}
\end{Highlighting}
\end{Shaded}

\begin{verbatim}
##    uno dos media
## 1   19  19  19.0
## 2   19  19  19.0
## 3   19  20  19.5
## 4   19  20  19.5
## 5   19  20  19.5
## 6   19  20  19.5
## 7   19  21  20.0
## 8   19  24  21.5
## 9   19  26  22.5
## 10  19  19  19.0
## 11  19  19  19.0
## 12  19  20  19.5
## 13  19  20  19.5
## 14  19  20  19.5
## 15  19  20  19.5
## 16  19  21  20.0
## 17  19  24  21.5
## 18  19  26  22.5
## 19  19  19  19.0
## 20  19  19  19.0
## 21  19  20  19.5
## 22  19  20  19.5
## 23  19  20  19.5
## 24  19  20  19.5
## 25  19  21  20.0
## 26  19  24  21.5
## 27  19  26  22.5
## 28  20  19  19.5
## 29  20  19  19.5
## 30  20  19  19.5
## 31  20  20  20.0
## 32  20  20  20.0
## 33  20  20  20.0
## 34  20  21  20.5
## 35  20  24  22.0
## 36  20  26  23.0
## 37  20  19  19.5
## 38  20  19  19.5
## 39  20  19  19.5
## 40  20  20  20.0
## 41  20  20  20.0
## 42  20  20  20.0
## 43  20  21  20.5
## 44  20  24  22.0
## 45  20  26  23.0
## 46  20  19  19.5
## 47  20  19  19.5
## 48  20  19  19.5
## 49  20  20  20.0
## 50  20  20  20.0
## 51  20  20  20.0
## 52  20  21  20.5
## 53  20  24  22.0
## 54  20  26  23.0
## 55  20  19  19.5
## 56  20  19  19.5
## 57  20  19  19.5
## 58  20  20  20.0
## 59  20  20  20.0
## 60  20  20  20.0
## 61  20  21  20.5
## 62  20  24  22.0
## 63  20  26  23.0
## 64  21  19  20.0
## 65  21  19  20.0
## 66  21  19  20.0
## 67  21  20  20.5
## 68  21  20  20.5
## 69  21  20  20.5
## 70  21  20  20.5
## 71  21  24  22.5
## 72  21  26  23.5
## 73  24  19  21.5
## 74  24  19  21.5
## 75  24  19  21.5
## 76  24  20  22.0
## 77  24  20  22.0
## 78  24  20  22.0
## 79  24  20  22.0
## 80  24  21  22.5
## 81  24  26  25.0
## 82  26  19  22.5
## 83  26  19  22.5
## 84  26  19  22.5
## 85  26  20  23.0
## 86  26  20  23.0
## 87  26  20  23.0
## 88  26  20  23.0
## 89  26  21  23.5
## 90  26  24  25.0
\end{verbatim}

\begin{Shaded}
\begin{Highlighting}[]
\NormalTok{x=}\KeywordTok{sort}\NormalTok{(}\KeywordTok{unique}\NormalTok{(casos\_edad}\OperatorTok{$}\NormalTok{media))}
\NormalTok{x}
\end{Highlighting}
\end{Shaded}

\begin{verbatim}
##  [1] 19.0 19.5 20.0 20.5 21.5 22.0 22.5 23.0 23.5 25.0
\end{verbatim}

\begin{Shaded}
\begin{Highlighting}[]
\NormalTok{CF=}\KeywordTok{table}\NormalTok{(casos\_edad}\OperatorTok{$}\NormalTok{media)}

\NormalTok{CF}
\end{Highlighting}
\end{Shaded}

\begin{verbatim}
## 
##   19 19.5   20 20.5 21.5   22 22.5   23 23.5   25 
##    6   24   18    8    6    8    8    8    2    2
\end{verbatim}

\begin{Shaded}
\begin{Highlighting}[]
\NormalTok{probs=}\KeywordTok{prop.table}\NormalTok{(}\KeywordTok{table}\NormalTok{(casos\_edad}\OperatorTok{$}\NormalTok{media))}
\NormalTok{probs}
\end{Highlighting}
\end{Shaded}

\begin{verbatim}
## 
##         19       19.5         20       20.5       21.5         22       22.5 
## 0.06666667 0.26666667 0.20000000 0.08888889 0.06666667 0.08888889 0.08888889 
##         23       23.5         25 
## 0.08888889 0.02222222 0.02222222
\end{verbatim}

\begin{Shaded}
\begin{Highlighting}[]
\NormalTok{sol\_df=}\KeywordTok{data.frame}\NormalTok{(}\DataTypeTok{Media\_Edad=}\NormalTok{x,}\DataTypeTok{Freq\_Absolutas=}\KeywordTok{as.numeric}\NormalTok{(CF),}\DataTypeTok{Probababilidades=}\KeywordTok{as.numeric}\NormalTok{(probs))}
\NormalTok{knitr}\OperatorTok{::}\KeywordTok{kable}\NormalTok{(sol\_df)}
\end{Highlighting}
\end{Shaded}

\begin{longtable}[]{@{}rrr@{}}
\toprule
Media\_Edad & Freq\_Absolutas & Probababilidades\tabularnewline
\midrule
\endhead
19.0 & 6 & 0.0666667\tabularnewline
19.5 & 24 & 0.2666667\tabularnewline
20.0 & 18 & 0.2000000\tabularnewline
20.5 & 8 & 0.0888889\tabularnewline
21.5 & 6 & 0.0666667\tabularnewline
22.0 & 8 & 0.0888889\tabularnewline
22.5 & 8 & 0.0888889\tabularnewline
23.0 & 8 & 0.0888889\tabularnewline
23.5 & 2 & 0.0222222\tabularnewline
25.0 & 2 & 0.0222222\tabularnewline
\bottomrule
\end{longtable}

\hypertarget{problema-2.}{%
\subsection{Problema 2.}\label{problema-2.}}

Verificar que: \[F_W (t)=
\left\{\begin{array}{ll}
0, & \mbox{si $t<3$},
 \\[0.1cm]
{1\over 3}, & \mbox{si $3\leq t<4$},
 \\[0.1cm]
{1\over 2}, & \mbox{si $4\leq t<5$},
 \\[0.1cm] 
{2\over 3}, & \mbox{si $5\leq t<6$},
 \\[0.1cm] 
1, & \mbox{si $t\geq 6$},
\end{array}\right.
\] es una función de distribución y especificar la función de
probabilidad para \(W\). Hallar también \(P(3<W\leq 5)\).

\hypertarget{soluciuxf3n-1}{%
\subsubsection{Solución}\label{soluciuxf3n-1}}

\begin{Shaded}
\begin{Highlighting}[]
\NormalTok{FX=}\ControlFlowTok{function}\NormalTok{(x)\{}
\NormalTok{  aux=}\ControlFlowTok{function}\NormalTok{(t)\{}
    \ControlFlowTok{if}\NormalTok{(t}\OperatorTok{\textless{}}\DecValTok{3}\NormalTok{) \{}\KeywordTok{return}\NormalTok{(}\DecValTok{0}\NormalTok{)\}}
    \ControlFlowTok{if}\NormalTok{(}\DecValTok{3}\OperatorTok{\textless{}=}\NormalTok{t }\OperatorTok{\&}\StringTok{ }\NormalTok{t}\OperatorTok{\textless{}}\DecValTok{4}\NormalTok{) \{}\KeywordTok{return}\NormalTok{(}\DecValTok{1}\OperatorTok{/}\DecValTok{3}\NormalTok{)\}}
    \ControlFlowTok{if}\NormalTok{(}\DecValTok{4}\OperatorTok{\textless{}=}\StringTok{ }\NormalTok{t }\OperatorTok{\&}\StringTok{ }\NormalTok{t}\OperatorTok{\textless{}}\StringTok{ }\DecValTok{5}\NormalTok{) \{}\KeywordTok{return}\NormalTok{(}\DecValTok{1}\OperatorTok{/}\DecValTok{2}\NormalTok{)\}}
    \ControlFlowTok{if}\NormalTok{(}\DecValTok{5}\OperatorTok{\textless{}=}\StringTok{ }\NormalTok{t }\OperatorTok{\&}\StringTok{ }\NormalTok{t}\OperatorTok{\textless{}}\StringTok{ }\DecValTok{6}\NormalTok{) \{}\KeywordTok{return}\NormalTok{(}\DecValTok{2}\OperatorTok{/}\DecValTok{3}\NormalTok{)\}}
    \ControlFlowTok{if}\NormalTok{(t}\OperatorTok{\textgreater{}=}\DecValTok{6}\NormalTok{)\{}\KeywordTok{return}\NormalTok{(}\DecValTok{1}\NormalTok{)\}}
\NormalTok{    \}}
  \KeywordTok{sapply}\NormalTok{(x,}\DataTypeTok{FUN=}\NormalTok{aux)}
\NormalTok{\}}

\KeywordTok{curve}\NormalTok{(FX,}\DecValTok{0}\NormalTok{,}\DecValTok{7}\NormalTok{,}\DataTypeTok{col=}\StringTok{"blue"}\NormalTok{)}
\end{Highlighting}
\end{Shaded}

\includegraphics{Tema-2---Variables-Aleatorias_Soluciones_files/figure-latex/unnamed-chunk-2-1.pdf}

La función \(F_X\) cumple todas las propiedades de una función de
distribución discreta:

\begin{itemize}
\tightlist
\item
  \(0\leq F_X(t)\leq 1\) para todo \(t\in \mathbb{R}.\)
\item
  Es solo continua por la derecha, luego es dicreta no es continua con
  dominio \(D_X=\{3,4,5,6\}\) que son los valores dónde
  \(P(X=x)=F_X(x)-F_X(x-)\not=0\).
\item
  Tiende asintóticamente a 1 cuando \(x\to+\infty\) y a 0 cuandor
  \(x\to-\infty\).
\end{itemize}

El Dominio es \(D_X=\{3,4,5,6\}\)

\(P(X=3)=F_X(3)-F_X(3^{-})=F_X(3)-lim_{x\to 3^{-}} F_X(x)=\frac{1}{3}=\frac{1}{3}-0=\frac{1}{3}.\)

\(P(X=4)=F_X(4)-F_X(4^{-})=F_X(4)-lim_{x\to 4^{-}} F_X(x)=\frac{1}{2}-\frac{1}{3}=\frac{1}{6}.\)

\(P(X=5)=F_X(5)-F_X(5^{-})=F_X(5)-lim_{x\to 5^{-}} F_X(x)=\frac{2}{3}-\frac{1}{2}=\frac{1}{6}.\)

\(P(X=6)=F_X(6)-F_X(6^{-})=F_X(6)-lim_{x\to 5^{-}} F_X(x)=1-\frac{2}{3}=\frac{1}{3}.\)

\(P(X=x)=0\) si \(x \not\in\{3,4,5,6\}.\)

\hypertarget{problema-3.}{%
\subsection{Problema 3.}\label{problema-3.}}

La variable aleatoria \(Z\) tiene por función de probabilidad:
\[f_Z (x)=
\left\{\begin{array}{ll}
{1\over 3}, & \mbox{si $x=0,1,2$},\\ 0, & \mbox{en los otros
casos.}
\end{array}\right.
\] ¿Cuál es la función de distribución para \(Z\)?

\hypertarget{soluciuxf3n-2}{%
\subsubsection{Solución}\label{soluciuxf3n-2}}

Es discreta.

\hypertarget{problema-4.}{%
\subsection{Problema 4.}\label{problema-4.}}

Sea \(X_n\) una variable aleatoria dependiendo de un valor natural \(n\)
cuya función de probabilidad es: \[
f(x)=\begin{cases}
k\cdot i, & \mbox{si }i=1,2\ldots,n, \\
0, & \mbox{en caso contrario.}
\end{cases}
\] - Hallar el valor de \(k\) y la función de distribución de \(X\). -
Calcular la probabilidad de que \(X\) tome un valor par.

\hypertarget{soluciuxf3n-3}{%
\subsubsection{Solución}\label{soluciuxf3n-3}}

\hypertarget{problema-5.}{%
\subsection{Problema 5.}\label{problema-5.}}

Un examen tipo test consta de cinco preguntas con tres posibles opciones
cada una. Un alumno contesta al azar las cinco cuestiones. Suponiendo
que cada respuesta acertada vale dos puntos, hallar la distribución de
número de puntos obtenidos por el alumno.

\hypertarget{soluciuxf3n-4}{%
\subsubsection{Solución}\label{soluciuxf3n-4}}

\hypertarget{problema-6.}{%
\subsection{Problema 6.}\label{problema-6.}}

Continuamos con el ejercicio anterior pero ahora suponemos que restamos
una cierta cantidad por respuesta incorrecta. Suponiendo que el examen
tiene \(n\) preguntas, cada pregunta tiene \(k\) posibles respuestas, y
que cada pregunta acertada vale 1 punto, ¿qué cantidad hay que restar a
cada pregunta para que la esperanza de la nota de una pregunta
contestada al azar sea 0? Repetir el ejercicio anterior pero ahora
suponiendo que restamos a cada pregunta la cantidad obtenida en el caso
en que éste se reponda de forma errónea.

\hypertarget{soluciuxf3n-5}{%
\subsubsection{Solución}\label{soluciuxf3n-5}}

\hypertarget{problema-6.-1}{%
\subsection{Problema 6.}\label{problema-6.-1}}

Hallar la esperanza y la varianza de todas las variables que han
aparecido en los ejercicios anteriores.

\hypertarget{variables-aleatorias-continuas}{%
\section{Variables aleatorias
continuas}\label{variables-aleatorias-continuas}}

\hypertarget{problema-1.-1}{%
\subsection{Problema 1.}\label{problema-1.-1}}

Verificar que: \[
F_X (t)=
\left\{\begin{array}{ll}
0, & \mbox{si $t<-1$},
 \\[0.1cm]
{t+1\over 2}, & \mbox{si $-1\leq
t\leq 1$},
 \\[0.1cm]
1, & \mbox{si $t> 1$},
\end{array}\right.
\] es una función de distribución y hallar la función de densidad para
\(X\). Calcular también
\(P\left(-{1\over 2}\leq X\leq {1\over 2}\right)\).

\hypertarget{soluciuxf3n-6}{%
\subsubsection{Solución}\label{soluciuxf3n-6}}

\hypertarget{problema-2.-1}{%
\subsection{Problema 2.}\label{problema-2.-1}}

Sea \(Y\) una variable continua con función de densidad: \[
f_Y(y)=
\left\{\begin{array}{ll}
2(1-y), & \mbox{si $0<y<1$},\\ 0, & \mbox{en los otros casos}.
\end{array}\right.
\] Hallar la función de distribución \(F_Y(t)\).

\hypertarget{soluciuxf3n-7}{%
\subsubsection{Solución}\label{soluciuxf3n-7}}

\hypertarget{problema-3.-1}{%
\subsection{Problema 3.}\label{problema-3.-1}}

Verificar que: \[
F_Y(t)=
\left\{\begin{array}{ll}
0, & \mbox{si $t<0$},\\
\sqrt{t}, & \mbox{si $0\leq t\leq 1$},\\ 1, &
\mbox{si $t>1$},
\end{array}\right.
\] es una función de distribución y especificar la función de densidad
para \(Y\). Usar este resultado para hallar
\(P\left(-{1\over 2}<Y<{3\over 4}\right)\).

\hypertarget{soluciuxf3n-8}{%
\subsubsection{Solución}\label{soluciuxf3n-8}}

\hypertarget{problema-4.-1}{%
\subsection{Problema 4.}\label{problema-4.-1}}

Sea \(X\) una variable aleatoria con función de densidad: \[
f(x)=\begin{cases}
1-|x|, & \mbox{si }|x|\leq 1,\\
0, & \mbox{en caso contrario.}
\end{cases}
\] 1. Representar gráficamente dicha función. 2. Hallar y dibujar la
función de distribución. 3. Calcular las siguientes probabilidades:
\(P(X\geq 0)\) y \(P\left(|X|<\frac{1}{2}\right).\)

\hypertarget{soluciuxf3n-9}{%
\subsubsection{Solución}\label{soluciuxf3n-9}}

\hypertarget{problema-5.-1}{%
\subsection{Problema 5.}\label{problema-5.-1}}

Hallar la esperanza y la varianza de las variables de los ejercicios
anteriores.

\hypertarget{soluciuxf3n-10}{%
\subsubsection{Solución}\label{soluciuxf3n-10}}

\hypertarget{transformaciuxf3n-de-variables-aleatorias}{%
\section{Transformación de variables
aleatorias}\label{transformaciuxf3n-de-variables-aleatorias}}

\hypertarget{problema-1.-2}{%
\subsection{Problema 1.}\label{problema-1.-2}}

A partir de \[
F_X(t)=
\left\{\begin{array}{ll}
0, & \mbox{si $t<-1$},
\\[0.1cm]
{t+1\over 2}, & \mbox{si $-1\leq t\leq
1$},
 \\[0.1cm]
1, & \mbox{si $t>1$},
\end{array}\right.
\]

hallar la función de distribución para \(Y=15+2X\) y la función de
densidad para \(Y\).

\hypertarget{soluciuxf3n-11}{%
\subsubsection{Solución}\label{soluciuxf3n-11}}

Como \(D_X=[-1,1]\) entonces \(Y=15+2 X\) varía desde
\(Y=15+2\cdot (-1)=13\) hasta \(Y=15+2\cdot 1=17\) y por lo tanto su
dominio es \(D_Y=[13,17].\)

\begin{eqnarray*}
F_Y(y)&=& P(Y\leq y) =P(15+2\cdot X\leq y )=P(X\leq \frac{y-15}{2})
\\&=& 
F_X(y\frac{y-15}{2})=
\left\{\begin{array}{ll}
0, & \mbox{si } \frac{y-15}{2}<-1,
\\[0.1cm]
{\frac{y-15}{2}+1\over 2}, & \mbox{si } -1\leq y \leq
1,
 \\[0.1cm]
1, & \mbox{si } \frac{y-15}{2}>1,
\end{array}\right.
\\&=& 
\left\{\begin{array}{ll}
0, & \mbox{si } y<-2-15=13
,
\\[0.1cm]
\frac{y-13}{4}, & \mbox{si } 13\leq y \leq
17,
 \\[0.1cm]
1, & \mbox{si } y>17,
\end{array}\right.
\end{eqnarray*}

\[
f_Y(y)=(F_Y(y))'=
\left\{\begin{array}{ll}
\frac{1}{4}, & \mbox{si } 13\leq y \leq
17,
 \\[0.1cm]
0, & \mbox{ en cualquier otro caso}.
\end{array}\right.
\]

\hypertarget{problema-2.-2}{%
\subsection{Problema 2.}\label{problema-2.-2}}

Si \[
F_X(t)=
\left\{\begin{array}{ll}
0, & \mbox{si $t<0$},\\ t, &
\mbox{si $0\leq t\leq 1$},\\ 1, & \mbox{si
$t>1$},
\end{array}\right.
\] hallar la función de distribución y la función de densidad de la
forma estándar de \(X\) (\(Z={X-\mu_X\over \sigma_X}\)), donde
\(\mu_X =E(X)\) y \(\sigma_X=\sqrt{\mathrm{Var}(X)}\).

\hypertarget{soluciuxf3n-12}{%
\subsubsection{Solución}\label{soluciuxf3n-12}}

\hypertarget{problema-3.-2}{%
\subsection{Problema 3.}\label{problema-3.-2}}

Para formar un jardín circular, un jardinero corta una cuerda, la ata a
una estaca y marca el perímetro. Suponer que la longitud de la cuerda
tiene la misma verosimilitud de estar en el intervalo comprendido entre
\(r-0.1\) y \(r+0.1\). ¿Cuál es la distribución de \(X\), el área de la
superficie del jardín? ¿Cuál es la probabilidad de que el área de la
superficie sea mayor que \(\pi r^2\)?

\hypertarget{soluciuxf3n-13}{%
\subsubsection{Solución}\label{soluciuxf3n-13}}

\hypertarget{problema-4.-2}{%
\subsection{Problema 4.}\label{problema-4.-2}}

Sea \(X\) una variable aleatoria continua con función de densidad
\(f_X(x)\). Consideramos la variable aleatoria \(Y=\mathrm{e}^X\).
Hallar la función de densidad de la variable aleatoria \(Y\),
\(f_Y(y)\).

\hypertarget{soluciuxf3n-14}{%
\subsubsection{Solución}\label{soluciuxf3n-14}}

\end{document}
