% Options for packages loaded elsewhere
\PassOptionsToPackage{unicode}{hyperref}
\PassOptionsToPackage{hyphens}{url}
\PassOptionsToPackage{dvipsnames,svgnames*,x11names*}{xcolor}
%
\documentclass[
]{article}
\usepackage{lmodern}
\usepackage{amssymb,amsmath}
\usepackage{ifxetex,ifluatex}
\ifnum 0\ifxetex 1\fi\ifluatex 1\fi=0 % if pdftex
  \usepackage[T1]{fontenc}
  \usepackage[utf8]{inputenc}
  \usepackage{textcomp} % provide euro and other symbols
\else % if luatex or xetex
  \usepackage{unicode-math}
  \defaultfontfeatures{Scale=MatchLowercase}
  \defaultfontfeatures[\rmfamily]{Ligatures=TeX,Scale=1}
\fi
% Use upquote if available, for straight quotes in verbatim environments
\IfFileExists{upquote.sty}{\usepackage{upquote}}{}
\IfFileExists{microtype.sty}{% use microtype if available
  \usepackage[]{microtype}
  \UseMicrotypeSet[protrusion]{basicmath} % disable protrusion for tt fonts
}{}
\makeatletter
\@ifundefined{KOMAClassName}{% if non-KOMA class
  \IfFileExists{parskip.sty}{%
    \usepackage{parskip}
  }{% else
    \setlength{\parindent}{0pt}
    \setlength{\parskip}{6pt plus 2pt minus 1pt}}
}{% if KOMA class
  \KOMAoptions{parskip=half}}
\makeatother
\usepackage{xcolor}
\IfFileExists{xurl.sty}{\usepackage{xurl}}{} % add URL line breaks if available
\IfFileExists{bookmark.sty}{\usepackage{bookmark}}{\usepackage{hyperref}}
\hypersetup{
  pdftitle={Ejercicios Tema 0 - Repaso de combinatoria y conjuntos},
  pdfauthor={Ricardo Alberich, Juan Gabriel Gomila y Arnau Mir},
  colorlinks=true,
  linkcolor=Maroon,
  filecolor=Maroon,
  citecolor=Blue,
  urlcolor=Blue,
  pdfcreator={LaTeX via pandoc}}
\urlstyle{same} % disable monospaced font for URLs
\usepackage[margin=1in]{geometry}
\usepackage{graphicx,grffile}
\makeatletter
\def\maxwidth{\ifdim\Gin@nat@width>\linewidth\linewidth\else\Gin@nat@width\fi}
\def\maxheight{\ifdim\Gin@nat@height>\textheight\textheight\else\Gin@nat@height\fi}
\makeatother
% Scale images if necessary, so that they will not overflow the page
% margins by default, and it is still possible to overwrite the defaults
% using explicit options in \includegraphics[width, height, ...]{}
\setkeys{Gin}{width=\maxwidth,height=\maxheight,keepaspectratio}
% Set default figure placement to htbp
\makeatletter
\def\fps@figure{htbp}
\makeatother
\setlength{\emergencystretch}{3em} % prevent overfull lines
\providecommand{\tightlist}{%
  \setlength{\itemsep}{0pt}\setlength{\parskip}{0pt}}
\setcounter{secnumdepth}{5}

\title{Ejercicios Tema 0 - Repaso de combinatoria y conjuntos}
\author{Ricardo Alberich, Juan Gabriel Gomila y Arnau Mir}
\date{Curso de Probabilidad y Variables Aleatorias con R y Python}

\begin{document}
\maketitle

{
\hypersetup{linkcolor=blue}
\setcounter{tocdepth}{2}
\tableofcontents
}
\hypertarget{ejercicios-de-combinatoria-1}{%
\section{Ejercicios de Combinatoria
1}\label{ejercicios-de-combinatoria-1}}

Breves soluciones de los problemas

\hypertarget{problema-1.}{%
\subsection{Problema 1.}\label{problema-1.}}

En una carrera en la que participan diez caballos ¿de cuántas maneras
diferentes se pueden dar los cuatro primeros lugares?

\hypertarget{problema-2}{%
\subsection{Problema 2}\label{problema-2}}

Una empresa de reciente creación encarga a un diseñador gráfico la
elaboración del su logotipo, indicando que ha de seleccionar exactamente
tres colores de una lista de seis. ¿Cuántos grupos tienen para elegir el
diseñador?

\hypertarget{problema-3.}{%
\subsection{Problema 3.}\label{problema-3.}}

¿Cuántas palabras diferentes, de cuatro letras, se pueden formar con la
palabra byte?

\hypertarget{problema-4.}{%
\subsection{Problema 4.}\label{problema-4.}}

¿De cuantas maneras diferentes se pueden elegir el director y el
subdirector de un departamento formado por 50 miembros?

\hypertarget{problema-5}{%
\subsection{Problema 5}\label{problema-5}}

Con once empleados ¿cuántos comités de empresa de cinco personas se
pueden formar?

\hypertarget{problema-6}{%
\subsection{Problema 6}\label{problema-6}}

¿Cuántas maneras distintas hay de colocar quince libros diferentes en
una estantería si queremos que el de Probabilidades esté el primero y el
de Estadística en el tercero?

\hypertarget{problema-7}{%
\subsection{Problema 7}\label{problema-7}}

¿Cuántos caracteres diferentes podemos formar utilizando a lo sumo a
tres símbolos de los utilizados en el alfabeto Morse?

\hypertarget{problema-8}{%
\subsection{Problema 8}\label{problema-8}}

Un supermercado organiza una rifa con un premio de una botella de cava
para todas las papeletas que tengan las dos últimas cifras iguales a las
correspondientes dos últimas cifras del número premiado en el sorteo de
Navidad. Supongamos que todos los décimos tienen cuatro cifras y que
existe un único décimo (participación) de cada numeración ¿Cuántas
botellas repartirá el supermercado?

\hypertarget{problema-9}{%
\subsection{Problema 9}\label{problema-9}}

¿Cuántas palabras diferentes podemos formar con todas las letras de la
palabra estadística?

\hypertarget{problema-10}{%
\subsection{Problema 10}\label{problema-10}}

En una tienda de regalos hay relojes de arena con cubetas de colores, y
no hay diferencia alguna entre las dos cubetas que forman cada reloj. Si
hay cuatro colores posibles y el color de los dos recipientes puede
coincidir ¿cuántos modelos de reloj de arena puede ofrecer el
establecimiento?

\hypertarget{problema-11}{%
\subsection{Problema 11}\label{problema-11}}

En una partida de parchís gana aquel jugador que consigue alcanzar antes
con sus cuatro fichas la llegada. Si hay cuatro jugadores y la partida
continua hasta que todos han completado el recorrido ¿cuántos órdenes de
llegadas hay para la dieciséis fichas?

\hypertarget{problema-12}{%
\subsection{Problema 12}\label{problema-12}}

Se han de repartir cinco becas entre diez españoles y seis extranjeros,
de manera que se den tres a españoles y dos a extranjeros ¿De cuántas
maneras se puede hacer el reparto?

\newpage

\hypertarget{ejercicios-de-combinatoria-2}{%
\section{Ejercicios de Combinatoria
2}\label{ejercicios-de-combinatoria-2}}

En todas las resoluciones se deja como ejercicio cómo generar todos los
objetos combinatorios con la librería \texttt{gtools} o de otra manera.

\hypertarget{problema-1}{%
\subsection{Problema 1}\label{problema-1}}

¿De cuantos modos diferentes se pueden colocar tres libros diferentes en
una estantería?

\hypertarget{problema-2-1}{%
\subsection{Problema 2}\label{problema-2-1}}

Seis personas entran en el cine. ¿De cuantos modos diferentes se pueden
sentar en una fila?

\hypertarget{problema-3}{%
\subsection{Problema 3}\label{problema-3}}

Tenemos tres premios diferentes para repartir entre una serie de
ciudadanos destacados. Si hay 4 candidatos a dichos premios, de cuantos
modos se pueden distribuir los premios en dos casos:

\begin{itemize}
\item
  \begin{enumerate}
  \def\labelenumi{\arabic{enumi}.}
  \tightlist
  \item
    Si un ciudadano puede recibir como máximo un premio
  \end{enumerate}
\item
  \begin{enumerate}
  \def\labelenumi{\arabic{enumi}.}
  \setcounter{enumi}{1}
  \tightlist
  \item
    Si un ciudadano puede recibir más de un premio.
  \end{enumerate}
\end{itemize}

\hypertarget{problema-4}{%
\subsection{Problema 4}\label{problema-4}}

Dado un conjunto de 15 puntos del plano, ¿cuantas líneas se necesitan
para juntar todos los posibles pares de puntos?

\hypertarget{problema-5-1}{%
\subsection{Problema 5}\label{problema-5-1}}

Dada una caja con los siguientes focos; 2 de 25 vatios, 4 de 40 vatios y
4 de 100 vatios, ¿de cuantos modos se pueden elegir 3 de ellos?

\hypertarget{problema-6-1}{%
\subsection{Problema 6}\label{problema-6-1}}

Supongamos que las placas de matrícula de coches se componen de tres
letras seguidas de tres dígitos. Si se pueden usar todas las
combinaciones posibles, ¿cuantas placas diferentes se podrían formar?

\hypertarget{problema-7-1}{%
\subsection{Problema 7}\label{problema-7-1}}

¿De cuantos modos diferentes se pueden enfrentar en un partido 2 equipos
de una liga que tenga 8?

\hypertarget{problema-8-1}{%
\subsection{Problema 8}\label{problema-8-1}}

En un almacén hay cajas rojas y verdes. - ¿De cuantas formas se pueden
colocar en fila 20 cajas si 15 son rojas y 5 son verdes? - ¿Y si hay 10
de cada color?

\end{document}
