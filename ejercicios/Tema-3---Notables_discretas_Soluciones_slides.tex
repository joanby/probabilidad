% Options for packages loaded elsewhere
\PassOptionsToPackage{unicode}{hyperref}
\PassOptionsToPackage{hyphens}{url}
%
\documentclass[
  ignorenonframetext,
]{beamer}
\usepackage{pgfpages}
\setbeamertemplate{caption}[numbered]
\setbeamertemplate{caption label separator}{: }
\setbeamercolor{caption name}{fg=normal text.fg}
\beamertemplatenavigationsymbolsempty
% Prevent slide breaks in the middle of a paragraph
\widowpenalties 1 10000
\raggedbottom
\setbeamertemplate{part page}{
  \centering
  \begin{beamercolorbox}[sep=16pt,center]{part title}
    \usebeamerfont{part title}\insertpart\par
  \end{beamercolorbox}
}
\setbeamertemplate{section page}{
  \centering
  \begin{beamercolorbox}[sep=12pt,center]{part title}
    \usebeamerfont{section title}\insertsection\par
  \end{beamercolorbox}
}
\setbeamertemplate{subsection page}{
  \centering
  \begin{beamercolorbox}[sep=8pt,center]{part title}
    \usebeamerfont{subsection title}\insertsubsection\par
  \end{beamercolorbox}
}
\AtBeginPart{
  \frame{\partpage}
}
\AtBeginSection{
  \ifbibliography
  \else
    \frame{\sectionpage}
  \fi
}
\AtBeginSubsection{
  \frame{\subsectionpage}
}
\usepackage{lmodern}
\usepackage{amssymb,amsmath}
\usepackage{ifxetex,ifluatex}
\ifnum 0\ifxetex 1\fi\ifluatex 1\fi=0 % if pdftex
  \usepackage[T1]{fontenc}
  \usepackage[utf8]{inputenc}
  \usepackage{textcomp} % provide euro and other symbols
\else % if luatex or xetex
  \usepackage{unicode-math}
  \defaultfontfeatures{Scale=MatchLowercase}
  \defaultfontfeatures[\rmfamily]{Ligatures=TeX,Scale=1}
\fi
% Use upquote if available, for straight quotes in verbatim environments
\IfFileExists{upquote.sty}{\usepackage{upquote}}{}
\IfFileExists{microtype.sty}{% use microtype if available
  \usepackage[]{microtype}
  \UseMicrotypeSet[protrusion]{basicmath} % disable protrusion for tt fonts
}{}
\makeatletter
\@ifundefined{KOMAClassName}{% if non-KOMA class
  \IfFileExists{parskip.sty}{%
    \usepackage{parskip}
  }{% else
    \setlength{\parindent}{0pt}
    \setlength{\parskip}{6pt plus 2pt minus 1pt}}
}{% if KOMA class
  \KOMAoptions{parskip=half}}
\makeatother
\usepackage{xcolor}
\IfFileExists{xurl.sty}{\usepackage{xurl}}{} % add URL line breaks if available
\IfFileExists{bookmark.sty}{\usepackage{bookmark}}{\usepackage{hyperref}}
\hypersetup{
  pdftitle={Soluciones ejercicos variables aleatorias discretas},
  pdfauthor={Ricardo Alberich, Juan Gabriel Gomila y Arnau Mir},
  hidelinks,
  pdfcreator={LaTeX via pandoc}}
\urlstyle{same} % disable monospaced font for URLs
\newif\ifbibliography
\usepackage{color}
\usepackage{fancyvrb}
\newcommand{\VerbBar}{|}
\newcommand{\VERB}{\Verb[commandchars=\\\{\}]}
\DefineVerbatimEnvironment{Highlighting}{Verbatim}{commandchars=\\\{\}}
% Add ',fontsize=\small' for more characters per line
\usepackage{framed}
\definecolor{shadecolor}{RGB}{248,248,248}
\newenvironment{Shaded}{\begin{snugshade}}{\end{snugshade}}
\newcommand{\AlertTok}[1]{\textcolor[rgb]{0.94,0.16,0.16}{#1}}
\newcommand{\AnnotationTok}[1]{\textcolor[rgb]{0.56,0.35,0.01}{\textbf{\textit{#1}}}}
\newcommand{\AttributeTok}[1]{\textcolor[rgb]{0.77,0.63,0.00}{#1}}
\newcommand{\BaseNTok}[1]{\textcolor[rgb]{0.00,0.00,0.81}{#1}}
\newcommand{\BuiltInTok}[1]{#1}
\newcommand{\CharTok}[1]{\textcolor[rgb]{0.31,0.60,0.02}{#1}}
\newcommand{\CommentTok}[1]{\textcolor[rgb]{0.56,0.35,0.01}{\textit{#1}}}
\newcommand{\CommentVarTok}[1]{\textcolor[rgb]{0.56,0.35,0.01}{\textbf{\textit{#1}}}}
\newcommand{\ConstantTok}[1]{\textcolor[rgb]{0.00,0.00,0.00}{#1}}
\newcommand{\ControlFlowTok}[1]{\textcolor[rgb]{0.13,0.29,0.53}{\textbf{#1}}}
\newcommand{\DataTypeTok}[1]{\textcolor[rgb]{0.13,0.29,0.53}{#1}}
\newcommand{\DecValTok}[1]{\textcolor[rgb]{0.00,0.00,0.81}{#1}}
\newcommand{\DocumentationTok}[1]{\textcolor[rgb]{0.56,0.35,0.01}{\textbf{\textit{#1}}}}
\newcommand{\ErrorTok}[1]{\textcolor[rgb]{0.64,0.00,0.00}{\textbf{#1}}}
\newcommand{\ExtensionTok}[1]{#1}
\newcommand{\FloatTok}[1]{\textcolor[rgb]{0.00,0.00,0.81}{#1}}
\newcommand{\FunctionTok}[1]{\textcolor[rgb]{0.00,0.00,0.00}{#1}}
\newcommand{\ImportTok}[1]{#1}
\newcommand{\InformationTok}[1]{\textcolor[rgb]{0.56,0.35,0.01}{\textbf{\textit{#1}}}}
\newcommand{\KeywordTok}[1]{\textcolor[rgb]{0.13,0.29,0.53}{\textbf{#1}}}
\newcommand{\NormalTok}[1]{#1}
\newcommand{\OperatorTok}[1]{\textcolor[rgb]{0.81,0.36,0.00}{\textbf{#1}}}
\newcommand{\OtherTok}[1]{\textcolor[rgb]{0.56,0.35,0.01}{#1}}
\newcommand{\PreprocessorTok}[1]{\textcolor[rgb]{0.56,0.35,0.01}{\textit{#1}}}
\newcommand{\RegionMarkerTok}[1]{#1}
\newcommand{\SpecialCharTok}[1]{\textcolor[rgb]{0.00,0.00,0.00}{#1}}
\newcommand{\SpecialStringTok}[1]{\textcolor[rgb]{0.31,0.60,0.02}{#1}}
\newcommand{\StringTok}[1]{\textcolor[rgb]{0.31,0.60,0.02}{#1}}
\newcommand{\VariableTok}[1]{\textcolor[rgb]{0.00,0.00,0.00}{#1}}
\newcommand{\VerbatimStringTok}[1]{\textcolor[rgb]{0.31,0.60,0.02}{#1}}
\newcommand{\WarningTok}[1]{\textcolor[rgb]{0.56,0.35,0.01}{\textbf{\textit{#1}}}}
\setlength{\emergencystretch}{3em} % prevent overfull lines
\providecommand{\tightlist}{%
  \setlength{\itemsep}{0pt}\setlength{\parskip}{0pt}}
\setcounter{secnumdepth}{-\maxdimen} % remove section numbering

\title{Soluciones ejercicos variables aleatorias discretas}
\author{Ricardo Alberich, Juan Gabriel Gomila y Arnau Mir}
\date{}

\begin{document}
\frame{\titlepage}

\hypertarget{ejercicios-distribuciones-notables-discretas}{%
\section{Ejercicios distribuciones notables
discretas}\label{ejercicios-distribuciones-notables-discretas}}

\begin{frame}{Ejercicio 1}
\protect\hypertarget{ejercicio-1}{}
Se lanzan a la vez 5 dados (de parchís) bien balanceados. Sea \(X\) el
número\\
de unos que se observan en la cara superior del dado. Calcular la
esperanza de \(X\), la varianza de \(X\), \(P(1\leq X<4)\) y
\(P(X\geq 2).\)
\end{frame}

\begin{frame}{Ejercicio 1 solución}
\protect\hypertarget{ejercicio-1-soluciuxf3n}{}
La variable \(X=\)número de unos en el lanzamiento de 5 dados, es una
variable binomial \(B(n=5,p=1/6)\).

Así que su valor esperado es
\(E(X)=n\cdot p = 5\cdot \frac{1}{6}=\frac56\) y su varianza es
\(Var(X)=n\cdot p\cdot (1-p)= 5\cdot \frac16\cdot \frac56=\frac{25}{36}.\)
\end{frame}

\begin{frame}{Ejercicio 1 solución}
\protect\hypertarget{ejercicio-1-soluciuxf3n-1}{}
\[
\begin{array}{lll}
P(1\leq X< 4)&=&P(X<4)-P(X<1)=P(X\leq 3)-P(X=0)\\
&=&\sum_{x=0}^3 P(X=x)-P(X=0)\\
&=&
P(X=3)+P(X=2)+P(X=1)=
\\
&=& {5 \choose 3} \left(\frac16\right)^3 \left(\frac56\right)^2+
{5 \choose 2} \left(\frac16\right)^2 \left(\frac56\right)^3+
{5 \choose 1} \left(\frac16\right)^1 \left(\frac56\right)^4
\\
&=&
\frac{5!}{3!\cdot (5-3)!} \frac{5^2}{6^5}+
\frac{5!}{2!\cdot (5-2)!} \frac{5^3}{6^5}+
\frac{5!}{1!\cdot (5-1)!} \frac{5^1}{6^5} \\
&=&
10 \frac{5^2}{6^5}+
10 \frac{5^3}{6^5}+
5 \frac{5^4}{6^5}=\frac{10\cdot 25 +10 \cdot 125 +5\cdot 625}{776}\\&=&
\frac{4625}{7776}=0.5947788 
\end{array}
\]
\end{frame}

\begin{frame}[fragile]{Ejercicio 1 solución}
\protect\hypertarget{ejercicio-1-soluciuxf3n-2}{}
Con R

\[
\begin{array}{lll}
P(1\leq X< 4)&=& P(X<4)-P(X<1)=P(X\leq 3)-P(X=0)\\&=&0.5947788.
\end{array}
\]

\begin{Shaded}
\begin{Highlighting}[]
\KeywordTok{pbinom}\NormalTok{(}\DecValTok{3}\NormalTok{,}\DataTypeTok{size=}\DecValTok{5}\NormalTok{,}\DataTypeTok{prob=}\DecValTok{1}\OperatorTok{/}\DecValTok{6}\NormalTok{)}
\end{Highlighting}
\end{Shaded}

\begin{verbatim}
## [1] 0.9966564
\end{verbatim}

\begin{Shaded}
\begin{Highlighting}[]
\KeywordTok{dbinom}\NormalTok{(}\DecValTok{0}\NormalTok{,}\DataTypeTok{size=}\DecValTok{5}\NormalTok{,}\DataTypeTok{prob=}\DecValTok{1}\OperatorTok{/}\DecValTok{6}\NormalTok{)}
\end{Highlighting}
\end{Shaded}

\begin{verbatim}
## [1] 0.4018776
\end{verbatim}

\begin{Shaded}
\begin{Highlighting}[]
\KeywordTok{pbinom}\NormalTok{(}\DecValTok{3}\NormalTok{,}\DataTypeTok{size=}\DecValTok{5}\NormalTok{,}\DataTypeTok{prob=}\DecValTok{1}\OperatorTok{/}\DecValTok{6}\NormalTok{)}\OperatorTok{{-}}\KeywordTok{dbinom}\NormalTok{(}\DecValTok{0}\NormalTok{,}\DataTypeTok{size=}\DecValTok{5}\NormalTok{,}\DataTypeTok{prob=}\DecValTok{1}\OperatorTok{/}\DecValTok{6}\NormalTok{)}
\end{Highlighting}
\end{Shaded}

\begin{verbatim}
## [1] 0.5947788
\end{verbatim}
\end{frame}

\begin{frame}{Ejercicio 2}
\protect\hypertarget{ejercicio-2}{}
El 10\% de los usb fabricados por una marca tienen algún defecto (pero
son baratos). Si se seleccionan al azar 10 de los usb fabricados por
esta fábrica, ¿cuáles la probabilidad de que ninguno sea defectuoso?
¿Cuántos usb defectuosos debemos esperar?
\end{frame}

\begin{frame}{Ejercicio 2 solución}
\protect\hypertarget{ejercicio-2-soluciuxf3n}{}
Bajo estas condiciones y suponiendo independencia entre la probabilidad
de defecto, la variable \(X\)= número de usb defectuosos sigue una ley
\(B(n=10,p=0.1).\)

Nos piden \[
\small{
\begin{array}{lll}
P(\mbox{ningún defectuoso entre 10})&=& P(X=0)={10\choose 0} 0.1^0\cdot(1-0.1)^{10}=0.9^{10} \\&=&
0.3486784.
\end{array}
}
\]

El valor esperado es \(E(X)=n\cdot p =10\cdot 0.1=1.\)
\end{frame}

\begin{frame}{Ejercicio 3}
\protect\hypertarget{ejercicio-3}{}
Si \(Y\) sigue una distribución binomial con media \(\mu_Y=6\) y
varianza \(\sigma_Y^2=4\). Calcular la distribución de \(Y\), es decir,
encontrad los valores de \(n\) y \(p\).
\end{frame}

\begin{frame}{Ejercicio 3 solución}
\protect\hypertarget{ejercicio-3-soluciuxf3n}{}
Tenemos que \(Y\) es una \(B(n,p)\) luego

\[E(Y)=n\cdot p=\mu_Y=6\]

y

\[Var(Y)=n\cdot p\cdot (1-p)=\sigma_Y^2=4.\]

Ahora \(p=\frac{6}{n}\) y sustituyendo en la segunda igualdad
\(n\cdot \frac{6}{n} \cdot (1-\frac{6}{n})=4\); de donde
\(6 \left(1-\frac6n\right)= 4\), \(2 \cdot n =36\) y finalmente
\(n=18\). Sabiendo \(n\) podemos calcular ahora
\(p=\frac{6}{18}=\frac13.\)
\end{frame}

\begin{frame}{Ejercicio 4}
\protect\hypertarget{ejercicio-4}{}
Un fabricante de \emph{bombillas inteligentes} controladas por
\emph{Bluetooth} las vende a sus distribuidores en lotes de 20.
Supongamos que la probabilidad de que una bombillas inteligentes esté
defectuosa es del \(0.05\). + a) ¿Cuál es el número esperado de
bombillas defectuosas por paquete. + b) ¿Cuál es la probabilidad de que
un determinado lote no tenga ninguna bombilla defectuosa?
\end{frame}

\begin{frame}{Ejercicio 4 solución}
\protect\hypertarget{ejercicio-4-soluciuxf3n}{}
Suponiendo independencia entre la probabilidad de defecto del lote
\(X=\) número de bombilla defectuosas en un o te de 20 sigue una ley
\(B(n=20,p=0.05)\)

El valor esperado es \(E(X)=n\cdot p= 20\cdot 0.05=1.\)

\(P(X=0)=`dbinom(0,size=20,prob=0.05)`=0.3584859.\)
\end{frame}

\begin{frame}{Ejercicio 5}
\protect\hypertarget{ejercicio-5}{}
Una urna contiene 10 bolas, una de color negro y las demás blancas. Sea
\(Z\) el número de extracciones con reposición necesarias para extraer
la bola negra. ¿Cuál es la distribución de la variable \(Z\)?
\end{frame}

\begin{frame}{Ejercicio 5 solución}
\protect\hypertarget{ejercicio-5-soluciuxf3n}{}
La extracciones son con reposición, así en cada extracción la
probabilidad de extraer negra es \(p=\frac{1}{10}\).

La variables \(X\) tendrá una distribución geométrica \(Ge(p=0.1)\) con
dominio \(D_X=\{1,2,3,\ldots\}\) ya que se se cuanta la extracción en la
que sale negra y se acaba el experimento.

Su función de probabilidad es
\(P(X=x)=(1-p)^{x-1}\cdot p=0.9^x \cdot 0.1\) para \(x=1,2,3\ldots.\)

Ejercicio dar su función de distribución su valor esperado y su
varianza.
\end{frame}

\begin{frame}{Ejercicio 6}
\protect\hypertarget{ejercicio-6}{}
Se lanza una moneda al aire hasta que sale cara. Supongamos que cada
tirada es independiente de las otras y que la probabilidad de que salga
cara cada vez es \(p\).\\
+ a) Demostrar que la probabilidad de que hagan falta un número impar de
lanzamientos es \({p\over 1-q^2}\) donde \(q=1-p\). + b) Encontrar el
valor de \(p\) tal que la probabilidad de que necesitemos un número
impar de intentos sea \(0.6\). + c) ¿Existe un valor de \(p\) tal que la
probabilidad de que haga falta un número impar de intentos sea \(0.5\)?
\end{frame}

\begin{frame}{Ejercicio 6 solución}
\protect\hypertarget{ejercicio-6-soluciuxf3n}{}
Claramente \(X\) número de lanzamientos independientes hasta que salga
cara (incluiremos la cara) es un \(Ge(p)\).

Su función de probabilidad es \(P(X=x)=(1-p)^{x-1}\cdot p\) para
\(x=1,2,3\ldots.\)

La probabilidad de impar sera

\[
\small{
\sum_{k=0}^{+\infty}P(2\cdot k +1)=\sum_{k=0}^{+\infty} (1-p)^{2k+1-1}\cdot p=
p \cdot \sum_{k=0}^{+\infty} \left((1-p)^{2}\right)^k=p\cdot \frac{1}{1-(1-p)^2}= \frac{p}{1-q^2}.
}
\]
\end{frame}

\begin{frame}{Ejercicio 6 solución}
\protect\hypertarget{ejercicio-6-soluciuxf3n-1}{}
Nos piden \(p\) tal que \(p\cdot \frac{1}{1-(1-p)^2}=0.6\) operando
obtenemos que

\(p=0.6\cdot (1-(1-p)^2\) entonces \(p=0.6\left(1-1+2 p-p^2\right)\)
operando

\(0.6 p^2-0.2 p=0\) luego \(p \cdot (0.6 p -0.2)=0\).

Las soluciones son \(p=0\) que no es posible y \(p=\frac{1}{3}\) que es
la única solución.

Repitiendo la ecuación anterior para que la probabilidad de impar sea
\(0.5\) obtenemos que la único solución es \(p=0\) así que no es
posible.
\end{frame}

\begin{frame}{Ejercicio 7}
\protect\hypertarget{ejercicio-7}{}
Se ha observado que el aforo medio de vehículos en un determinado paso
de un camino rural es de 3 coches/hora. Suponer que los instantes en que
pasan automóviles son independientes. Sea \(X\) el número de coches que
pasas por este lugar en un intervalo de 20 minutos. Calcular \(P(X=0)\)
y \(P(X\geq 2)\).
\end{frame}

\begin{frame}{Ejercicio 7 solución}
\protect\hypertarget{ejercicio-7-soluciuxf3n}{}
Bajo estas condiciones la variable aleatoria \(X_t\) número coches en
\(t\) horas sólo puede seguir la distribución notable con promedio de
llegadas por hora \(\lambda=3\) y por lo tanto el proceso de Poisson
asociado \(X_t\) sigue una ley de probabilidad
\(Po(\lambda \cdot t =3\cdot t)\)

Como 20 minutos es un \(\frac13\) de hora luego \(X_{\frac13}\) es una
\(Po(3\cdot \frac13=1)\). Nos piden \(P(X_{\frac13}=0)\) y que
\(P\left(X_{\frac{1}{3}}\geq 2 \right)\). Sabemos que
\(P(X_{\frac13}=x)=\frac{1^x}{x!} e^{-1}\) para \(x=0,1,2,\ldots.\)

Así que \(P(X_{\frac13}=0)=\frac{(1)^0}{0!} e^{-1}=e^{-1}=0.3678794.\)
\end{frame}

\begin{frame}{Ejercicio 7 solución}
\protect\hypertarget{ejercicio-7-soluciuxf3n-1}{}
\[
\begin{array}{lll}
P\left(X_{\frac13}\geq 2 \right)&=&P\left(X_{\frac13}< 2 \right)=P\left(X_{\frac13}\leq 1\right)\\
&=&
\sum_{x=0}^1 \frac{1^x}{x!} e^{-1}
\\
&=&
\frac{1^0}{0!} e^{-1}+\frac{1^1}{1!} e^{-1}\\
&=&
e^{-1}\left(1+1\right)=2\cdot e^{-3}=0.1991483.
\end{array}
\]
\end{frame}

\begin{frame}[fragile]{Ejercicio 7 solución}
\protect\hypertarget{ejercicio-7-soluciuxf3n-2}{}
Con R

\begin{Shaded}
\begin{Highlighting}[]
\KeywordTok{dpois}\NormalTok{(}\DecValTok{0}\NormalTok{,}\DataTypeTok{lambda=}\DecValTok{1}\NormalTok{)}\CommentTok{\# P(X=1)}
\end{Highlighting}
\end{Shaded}

\begin{verbatim}
## [1] 0.3678794
\end{verbatim}

\begin{Shaded}
\begin{Highlighting}[]
\KeywordTok{ppois}\NormalTok{(}\DecValTok{1}\NormalTok{,}\DataTypeTok{lambda=}\DecValTok{1}\NormalTok{)}\CommentTok{\# P(X\textless{}=1)}
\end{Highlighting}
\end{Shaded}

\begin{verbatim}
## [1] 0.7357589
\end{verbatim}

\begin{Shaded}
\begin{Highlighting}[]
\DecValTok{1}\OperatorTok{{-}}\KeywordTok{ppois}\NormalTok{(}\DecValTok{1}\NormalTok{,}\DataTypeTok{lambda=}\DecValTok{1}\NormalTok{)}\CommentTok{\# 1{-}P(X\textless{}=1)=P(X\textgreater{}=2)}
\end{Highlighting}
\end{Shaded}

\begin{verbatim}
## [1] 0.2642411
\end{verbatim}

\begin{Shaded}
\begin{Highlighting}[]
\KeywordTok{ppois}\NormalTok{(}\DecValTok{1}\NormalTok{,}\DataTypeTok{lambda=}\DecValTok{1}\NormalTok{,}\DataTypeTok{lower.tail =} \OtherTok{FALSE}\NormalTok{)}\CommentTok{\# P(X\textgreater{}1)=P(X\textgreater{}=2)}
\end{Highlighting}
\end{Shaded}

\begin{verbatim}
## [1] 0.2642411
\end{verbatim}
\end{frame}

\end{document}
